\documentclass[9pt]{beamer}
\usepackage{graphicx} \usetheme{CambridgeUS} \usecolortheme{seahorse} \usepackage{mdframed}
\usepackage{listings}
\usepackage{xcolor}

\definecolor{codegreen}{rgb}{0,0.6,0}
\definecolor{codegray}{rgb}{0.5,0.5,0.5}
\definecolor{codepurple}{rgb}{0.58,0,0.82}
\definecolor{backcolour}{rgb}{0.95,0.95,0.92}

\lstdefinestyle{mystyle}{
    backgroundcolor=\color{backcolour},   
    commentstyle=\color{codegreen},
    keywordstyle=\color{magenta},
    numberstyle=\tiny\color{codegray},
    stringstyle=\color{codepurple},
    basicstyle=\ttfamily\footnotesize,
    breakatwhitespace=false,         
    breaklines=true,                 
    captionpos=b,                    
    keepspaces=true,                 
    numbers=left,                    
    numbersep=5pt,                  
    showspaces=false,                
    showstringspaces=false,
    showtabs=false,                  
    tabsize=2
}

\lstset{style=mystyle}

\title[PAKE und Honeywords]{BSP: PAKE und Täuschungs Kennwörter}
\author[Steve Meireles]{Student: Steve Meireles \and Tutor: Marjan Skrobot}
\date{2024}

\begin{document}

\frame{\titlepage}

\begin{frame}
\frametitle{Inhaltsverzeichnis}
\tableofcontents
\end{frame}

\begin{frame}
\frametitle{Wissenschaftliche Frage}
\section{Wissenschaftliche Frage}
Wie erkennt
man, ob eine Kennwort-Datei im Besitz von Eindringlingen ist und kann man
	gleichzeitig Phishing-Angriffe verhindern?
\end{frame}

\begin{frame}
\frametitle{Honeywords}
\section{Honeywords}
\begin{itemize}
	\item Falsche Passwörter zur Kennwort-Datei hinzufügen
	\item Honeychecker ist ein seperates und robustes System
	\item Honeychecker speichert die Postionen der richtegen Kennwörter
	\item Honeychecker alarmiert System Administrator beim Auslösen eines Honeywords
\end{itemize}
\end{frame}

\begin{frame}[fragile, plain]
\frametitle{PAKE Protokolle}
\section{PAKE}
	\begin{itemize}
		\item Sicher eine Lang-Zeit-Schlüssel mit einem geteileten
			schwachen Passwort
		\item Diese Protokolle ermöglichen das mit Hilfe von zyklischen
			Gruppen
	\end{itemize}
\end{frame}

\begin{frame}
\frametitle{SweetPAKE: Alice und Bob}
\section{SweetPAKE}
Lass Alice und Bob zwei Parteien sein
\begin{enumerate}
	\item Alice generiert öffentlicher Schlüssel und privater
	\item Alice schickt öffentlicher Schlüssel zu Bob
	\item Bob genereriert eine List von Lang-Zeit Schlüsseln
	\item Verschlüsselt jeden Schlüssel dieser Liste
	\item Bob schickt die Liste zur Alice
	\item Alice entschlüsselt jedes Element bis erfolgreich
	\item Alice schickt dann die Position des korrekten Kennworts zur Bob
	\item Bob schickt die Position zum Honeychecker
	\item Honeychecker checkt ob die Kennwort-Datei in Besitz von Eindringle ist
\end{enumerate}
\end{frame}

\begin{frame}
\frametitle{Technische Arbeit}
\section{Technische Arbeit}
	\begin{itemize}
		\item Implementation von SweetPAKE
		\item Muss schnell genug sein
	\end{itemize}
	Vier Hauptfunktionen:
	\begin{itemize}
		\item generate(): Erster Schritt des SweetPAKE-Protokolls
		\item encryption(): Verschlüsselungsschritt
		\item decryption(): Entschlüsselungsschritt
		\item retrieve\_key(): Schlüsselabrufschritt
	\end{itemize}
\end{frame}

\begin{frame}[fragile]
\frametitle{Technische Arbeit: Abschnitt}
\begin{lstlisting}[language=Python]
def gen(self):
        #gen function
        group = self.params.group
        self.rundom_exponent = group.rundom_exponent(self.entropy_f)
        self.y1_elem = group.Base1.exp(self.rundom_exponent)
        self.y2_elem = group.Base2.exp(self.rundom_exponent)
        Y2_elem = self.y2_elem.elementmult(group.password_to_hash(self.pw))

        #self.outbound_message = (self.y1+self.Y2) <-- apk
        y1_bytes = self.y1_elem.to_bytes()
        Y2_bytes = Y2_elem.to_bytes()
        self.outbound_message =  y1_bytes + Y2_bytes

        username_size = len(self.username).to_bytes()

        outbound_id_und_message = self.side + username_size + self.username + self.outbound_message

        return outbound_id_und_message
\end{lstlisting}
\end{frame}

\begin{frame}
\frametitle{Schlussfolgerung}
\section{Schlussfolgerung}
Wie erkennt man, ob eine Kennwort-Datei im Besitz von Eindringlingen ist und
	kann man gleichzeitig Phishing-Angriffe verhindern?
\begin{itemize}
	\item SweetPAKE ist eine gute Antwort
	\item Kombiminiert Stärken von PAKE und Honeywords
	\item Mit der Implementierung kann man dies austesten
\end{itemize}
\end{frame}

\begin{frame}[fragile]
\begin{center}
    \textbf{Thank you for your attention!}\\
\end{center}
\end{frame}


\end{document}
