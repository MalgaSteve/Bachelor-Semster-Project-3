\documentclass[conference, compsoc]{IEEEtran}
\usepackage{datetime}
\usepackage[style=ieee]{biblatex}
\usepackage{subfiles}
\usepackage{tabularx}
\usepackage{array}
\usepackage{amsfonts}
\usepackage{amsmath}
\usepackage{listings}
\usepackage{xcolor}
\usepackage{tikz}
\usepackage{algpseudocode}
\usetikzlibrary{shapes.geometric, arrows}

\definecolor{codegreen}{rgb}{0,0.6,0}
\definecolor{codegray}{rgb}{0.5,0.5,0.5}
\definecolor{codepurple}{rgb}{0.58,0,0.82}
\definecolor{backcolour}{rgb}{0.95,0.95,0.92}

\lstdefinestyle{mystyle}{
    backgroundcolor=\color{backcolour},   
    commentstyle=\color{codegreen},
    keywordstyle=\color{magenta},
    numberstyle=\tiny\color{codegray},
    stringstyle=\color{codepurple},
    basicstyle=\ttfamily\footnotesize,
    breakatwhitespace=false,         
    breaklines=true,                 
    captionpos=b,                    
    keepspaces=true,                 
    numbers=left,                    
    numbersep=5pt,                  
    showspaces=false,                
    showstringspaces=false,
    showtabs=false,                  
    tabsize=2
}

\lstset{style=mystyle}
\addbibresource{biblio.bib}

\begin{document}

\title{PAKE protocols and Decoy passwords\\
{\small \today~-~\currenttime}}

\author{
    \IEEEauthorblockA{Steve Meireles Lopes\\
    University of Luxembourg\\
    Email: steve.meireles.001@student.uni.lu}\\
    This report has been produced under the supervision of:\\
    \IEEEauthorblockA{Marjan Skrobot\\
    University of Luxembourg\\
    Email: marjan.skrobot@uni.lu}}

\maketitle

\begin{abstract}
The paper is a third-semester Bachelor Project, it answers the scientific
	question: How to detect if a password file is in possession of
	intruders and at the same time prevent phishing attacks? It answers by
	analyzing Honeywords and PAKE protocols. It concludes that a
	combination of both answers the question and as a result, it analyzes the
	SweetPAKE protocol. The project provides an implementation of
	honeywords generation methods and a variant of the SweetPAKE protocol
	named BeePAKE which includes the use of the PAPKE protocol.
\end{abstract}

\begin{IEEEkeywords}
Secure authentication, password security, PAKE, Honeywords
\end{IEEEkeywords}

\section{Introduction}
\subfile{sections/introduction}

\section{Honeywords}
\subfile{sections/honeywords}

\section{PAKE protocols}
\subfile{sections/PAKE}

\section{SweetPAKE}
\subfile{sections/sweetPAKE}

\section{Application}
\subfile{sections/application}

\section{Conclusion}
\subfile{sections/conclusion}

\printbibliography

\end{document}
