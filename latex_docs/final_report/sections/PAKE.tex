\documentclass[../main.tex]{subfiles}

\begin{document}

PAKE is an abbreviation for Password Authenticated Key Exchange, such protocols
make it possible to have a secure communication using a weak shared secret key.
As already mentioned in the introduction there are several PAKE protocols. In
this section we are going first to look at some basic mandatory knowledge
needed to understand these protocols by explaining Cyclic Groups and the
Diffie-Hellman protocol. After that we are going to look at two protocols in
detail: EKE and PAPKE. EKE is a rather simpler PAKE protocol and PAPKE is an
important protocol because in the following section the SweetPAKE protocol is
build upon exactly this protocol.

\subsection{Cyclic Groups}

\subsubsection{Group}
A group \(g\) is structure that consists of a set of elements and an operation \(\cdot\)
that satifies four propreties: 

Closure: \[\forall g, h \in G,  g \cdot h \in G\]
Identity element: \[\exists i \in G, \forall g \in G, i \cdot g = g =
g \cdot e\]
Associativity: \[\forall g_1, g_2, g_3 \in G, (g_1 \cdot g_2) \cdot g_3 = g_1
\cdot (g_2 \cdot g_3)\]
Inverses:\[\forall g \in G \exists h \in G, g \cdot h = e = h \cdot g\]

\subsubsection{Generator}
A generator \(g\) is an element of a cyclic group \(G\), such that \(g\) when repeatedly using
a group operation \(\cdot\) on itself, it can generate all elements of \(G\).

\subsubsection{Definition} A group \(G\) is cyclic if there exists \(g \in G\)
such that \(g\) is a generator.

Important note: If \(G\) is a cyclic group and has generator \(g\) then \[G =
\{a^n | n \in \mathbb{Z}\}\]


\subsection{Diffie Hellman}
\subsection{EKE}
\subsection{PAPKE}

\end{document}
