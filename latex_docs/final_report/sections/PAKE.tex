\documentclass[../main.tex]{subfiles}

\begin{document}

PAKE is an abbreviation for Password Authenticated Key Exchange, such protocols
make it possible to have secure communication using a weak shared secret key.
As already mentioned in the introduction there are several PAKE protocols. In
this section, we are going first to look at some basic mandatory knowledge
needed to understand these protocols by explaining Cyclic Groups and the
Diffie-Hellman protocol. After that we are going to look at two protocols in
detail: EKE and PAPKE. EKE is a rather simpler PAKE protocol and PAPKE is an
important protocol because, in the following section, the SweetPAKE protocol is built upon exactly this protocol.

\subsection{Cyclic Groups}
This subsection covers the math base for the upcoming section.

\subsubsection{Group}
A group \(g\) is a structure that consists of a set of elements and an operation \(\cdot\)
that satisfies four properties: 

Closure: \[\forall g, h \in G,  g \cdot h \in G\]
Identity element: \[\exists i \in G, \forall g \in G, i \cdot g = g =
g \cdot e\]
Associativity: \[\forall g_1, g_2, g_3 \in G, (g_1 \cdot g_2) \cdot g_3 = g_1
\cdot (g_2 \cdot g_3)\]
Inverses:\[\forall g \in G \exists h \in G, g \cdot h = e = h \cdot g\]

\subsubsection{Generator}
A generator \(g\) is an element of a cyclic group \(G\), such that \(g\) when repeatedly using
a group operation \(\cdot\) on itself, it can generate all elements of \(G\).

\subsubsection{Definition} A group \(G\) is cyclic if there exists \(g \in G\)
such that \(g\) is a generator.

Important note: If \(G\) is a cyclic group and has generator \(g\) then \[G =
\{a^n | n \in \mathbb{Z}\}\]

\subsubsection{Decisional Diffie-Hellman assumption}
The assumption states that having \(g^{a}\) and \(g^{b}\), it is hard to compute \(g^{ab}\).

\subsection{Diffie-Hellman Key Exchange}
The Diffie-Hellman key exchange method is named after Whitfield Diffie and Martin Hellman.
The method allows two parties to establish a secret key without prior knowledge by using
cyclic groups.

First, two parties Alice and Bob agree on a modulus \(p\) and a generator
\(g\). Both can be publicly known.

Second, Alice generates a random integer \(a\), then sends Bob 

\[A = g^a \% p\]

while Bob also generates a random integer \(b\) and sends

\[B = g^b \% p\]
Now, Alice computes

\[g^{ab} = B^a \% p\]
and Bob computes

\[g^{ab} = A^b \% p\]

Both never send their generated key \(a\) and \(b\) therefore it is very hard
to compute \(g{ab}\) for an eavesdropper according to Decisional
Diffie-Hellman assumption. 

\subsection{EKE: Encrypted Key Exchange}
The first PAKE protocol was introduced by Bellovin and Merrit called EKE
\cite{bellovin1992encrypted}. In this section, we are going to analyze how the
protocol works. It allows two parties to securely authenticate and communicate
to each other over an insecure channel using a shared secret key namely a
password.

Let Alice and Bob be two parties communicating with each other using the EKE
protocol. Both share a common secret key, here \(s\) for secret. Often long
before the protocols begin they agree on a modulus \(p\) and a generator
\(g\). Alice starts by generating a random integer \(a\) and computes \(g^a \%
p\) encrypted with the secret key \(s\). She then sends her id \(id_a\) in
plain, and her encrypted key.

\[id_a, s(g^a \% p)\]

Bob receives the message and also generates a random integer \(b\) and computes
\(g^b \% p\) and encrypts it with the secret key \(s\). He then decrypts the
message by Alice and computes the session key \(k\).

\[g^{ab} \% p\]

Bob, then generates what is called a "challenge" by Bellovin and Merrit, which serves
to check the message is modified between the two parties. The challenge is encrypted
with the computed session key \(k\) and sent together with \(s(g^b \% b)\).

\[s(g^b \% p), k(challenge_B)\]

Next, Alice generates the session key \(k\) by decrypting \(s(g^b \% p)\) and computing

\[g^{ab} \% p\]

She then generates her challenge, encrypts it together with Bobs' challenge
and sends it to Bob. 

\[k(challenge_A, challenge_B)\]

Bob, then decrypts the message and checks, if \(challenge_B\) was changed during
communication by an adversary. He sends \(challenge_A\) back to Alice.

\[k(challenge_A)\]

Alice verifies if \(challenge_A\) is still the same.

Now, having a basic understanding of  the idea of PAKE protocol, more complex protocols
using this idea can be explained.

\subsection{PAPKE}
PAPKE is an abbreviation of Password-Authenticated Public-Key Encryption. It is
introduced by Bradley, Camenisch, Jarecki, Lehmann, Neven and Xu with a thorough
study \cite{bradley2019password}. It focuses on generating long-term keys. It
tries to replace to currently often used method where a third party using
certificates has to be trusted to provide a secure end-to-end encryption. The
study also points out that PAPKE implies PAKE but not the other way around,
this proves that PAPKE is an improved primitive. They also suggest two schemes
PAPKE named PAPKE-IC and PAPKE-FO. In this analysis, PAPKE-FO is a two-round
PAKE protocol that is going to be looked at since it is the one that is
implemented and used (see section Application). 

The scheme splitted into three functions \(Gen\), \(Enc\), and \(Dec\).
Let Alice and Bob be two parties who want to generate long-term keys.
Setup: Both parties agree on a group \(G\) of prime order p and two generators
\(g_1\), \(g_2\). This protocol uses three hash functions, thus the setup
consists of: 

\begin{itemize}
	\item password \(pwd\)
	\item Group \(G\)
	\item Generators \(g_1\) and \(g_2\)
	\item \(H_0: \{0,1\}^* \to G\)
	\item \(H_1: G^3 \times \{0,1\}^n \to \mathbb{Z}_p^2\)
	\item \(H_2: G \to \{0,1\}^n\)
\end{itemize}

\subsubsection{\(Gen\):} Alice does the following computations:
Generates random integers in the range of \(G\) (The \(\gets{_R}\) stands for
random assignment):
\[x \gets{_R} \text{ } \mathbb{Z}_p\] 
Generates two elements of \(G\) using both generators \(g_1\), \(g_2\) and \(x\).
\[y_1 \gets \text{ } g_1^x\] 
\[y_2 \gets \text{ } g_2^x\] 
Then computes \(Y_2\):
\[Y_2 \gets \text{ } y_2 \cdot H_0(pwd)\]
The tuple \((y_1, Y_2)\) gives us the public key \(apk\):
\[apk \gets \text{ } (y_1, Y_2)\]
Alice stores the secret key \(sk\):
\[sk \gets (x, y_1, y_2)\]
Alice sends her \(id\) and the public key \(apk\) to Bob.

The \(Gen\) function can be defined as follow:
\[Gen(pwd) = (sk, apk)\] 

\subsubsection{\(Enc\)} Bob generates random long-term key \(k\):
\[k \gets{_R} \text{ } \{0,1\}^*\] 
After that Bob interprets the message \(apk = (y_1, Y_2)\) and computes: 
\[y_2 \gets Y_2 \cdot H_0(pwd)^{-1}\]
Then he gets a random element of \(G\) and computes 2 elements \(r_1\) and
\(r_2\) using \(H_1\):
\[R \gets{_R} G\]
\[(r_1, r_2) \gets H_1(R, y_1, y_2, k)\]
In the next step, he starts computing secrets. The first secret \(c_1\) using both
generator and \(r_1\), \(r_2\). The second secret using \(y_1^{r_1}\),
\(y_2^{r_2}\) and \(R\). The third secret by XORing the hash of \(R\) and \(k\)
\[c_1 \gets g_1^{r_1} g_2^{r_2}\]
\[c_2 \gets y_1^{r_1} y_2^{r_2} \cdot R\]
\[c_3 \gets H_2(R) \oplus k\]
\[c = (c_1, c_2, c_3)\]
Bob then sends his \(id\) and the computed secrets \(c\).

The \(Enc\) function can be defined as:
\[Enc(pwd, apk) = c\]

\subsubsection{\(Dec\)} Alice interprets the inbound message  and computes
\(R\), the long term k by xoring \(c_3\) with the hash of the computed \(R\).
She also computes \(r_1\) and \(r_2\) to later confirm that nobody tried to
imitate Bob.
\[(r_1, r_2) \gets H_1(R, y_1, y_2, k)\]
Alice then checks if an adversary tried to masquerade as Bob:
\[\text{checks if } c_1 = g_1^{r_1} g_2^{r_2}\]

The \(Dec\) function can be defined as:
\[Dec(pwd, c) = k\]

It is a rather more complex protocol but uses the same mathematical principles as
Diffie-Hellman and EKE protocol with additional steps to ensure to share
long-term key that can be used for encrypting multiple messages.
\end{document}
