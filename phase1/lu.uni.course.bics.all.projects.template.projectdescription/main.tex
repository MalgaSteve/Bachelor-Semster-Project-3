\documentclass[conference,compsoc]{IEEEtran}
\usepackage{datetime}
\usepackage{caption}
\usepackage{listings}
%\usepackage{xcolor}

\usepackage[backend=biber]{biblatex}
\addbibresource{sample.bib}

\hyphenation{op-tical net-works semi-conduc-tor}

\begin{document}

\title{Combining Password Security with PAKE\\
{\small \today~-~\currenttime}}
 
\author{\IEEEauthorblockN{Meireles Lopes Steve}
\IEEEauthorblockA{University of Luxembourg\\
Email: steve.meireles.001@student.uni.lu}
\\
{\bf This report has been produced under the supervision of:}\\
\IEEEauthorblockN{Skrobot Marjan}
\IEEEauthorblockA{University of Luxembourg\\
Email: marjan.skrobot@uni.lu}%
}

\maketitle

\begin{abstract}

In today's world, password security is very important. Most people have weak
	passwords which exposes them to brute force attacks. Furthermore
	malicious attackers also often use phishing attacks. These attacks
	expose a huge vulnerability and therefore a risk for the data of a lot
	of people. This project is to answer the scientific question: In the
	context of client-to-server authentication over an insecure network,
	how does one detect if a password file on the server side (containing
	usernames and passwords) was compromised by intruders and at the same
	time prevent phishing attacks on clients?

\end{abstract}

\IEEEpeerreviewmaketitle

\section{Main required competencies } 

\subsection{Scientific main required competencies}
For the scientific deliverable the needs to know the basics of cryptography 
and security such as password security, symmetric and assymetric encryption and
key exchange. In addition to that he requires math knowledge such as set theory
and math notations.

\subsection{Technical main required competencies}
The technical deliverable requires the student to know the same topics as for
the scientifc deliverable. Moreover he will have to be able to program in the 
programming language python.


\section{ A Scientific Deliverable 1 } 
A lot of enterprises use username and password logins to authenticate their
users or employees. Everybody has a few technologies that require them to
authenticate with their username and password. It is commonly known that this is
often exploited by attackers through weak passwords, brute forcing or even
pishing attacks. This brings us to the scientific question: In the context of
client-to-server authentication over an insecure network, how does one detect if a
password file on the server side (containing usernames and passwords) was
compromised by intruders and at the same time prevent phishing attacks on
clients?

To answer this question the deliverable will study password security, secure
storage of password and brute forcing attacks. It will also study the honeyword
paper from Riverst Juels \cite{juels2013honeywords} to find technologies to
enhance the security propereties. Furthermore, it looks into ways to enhance
current password authentication protocols from sending a hash of a user's
password over TLS to utilzing PAKE which prevents pishing attack. On top of all
that the paper is going to examine the SweetPAKE protocol from Arriage, Ryan
and Skrobot \cite{sweetpake} which uses the PAKE and enhances it with decoy
password which adds password file leakage deetection mechanism at the server
side.

\section{A Technical Deliverable 1 }

The technical deliverable will be an implementation of a decoy password
generator according to proposed protocols from the Honeyword paper
\cite{juels2013honeywords} of Riverst Jules and the SweetPAKE protocol
\cite{sweetpake} suggested from Arriaga, Ryan and Skrobot in the programming
language python.

This project is beneficial to the security world considering there are not a
lot of open-source PAKE implementations despite of its security value.
Combining PAKE with a decoy password generator which further enhances password
security does not exist in the current market.

The program will contain a password file which simulates a password file in the
real world. It needs to have the ability to generate decoy passwords in a way
that an intruder cannot guess the correct password. It will have an index file
which simulates a file that an extern server, named here the honeychecker,
would have. The program should be able to check if the password given by the
user is the correct one. The program should also use SweetPake protocol to
communicate between server and client.

To be able to achieve this program will use the open-source project
python-spake2 of warner as reference. It will use the standard library 
random and hashlib to encrypt and generate passwords.

\printbibliography

\clearpage

\end{document}
